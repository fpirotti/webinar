\documentclass[]{elsarticle} %review=doublespace preprint=single 5p=2 column
%%% Begin My package additions %%%%%%%%%%%%%%%%%%%
\usepackage[hyphens]{url}

  \journal{Science of Remote Sensing} % Sets Journal name


\usepackage{lineno} % add
\providecommand{\tightlist}{%
  \setlength{\itemsep}{0pt}\setlength{\parskip}{0pt}}

\usepackage{graphicx}
\usepackage{booktabs} % book-quality tables
%%%%%%%%%%%%%%%% end my additions to header

\usepackage[T1]{fontenc}
\usepackage{lmodern}
\usepackage{amssymb,amsmath}
\usepackage{ifxetex,ifluatex}
\usepackage{fixltx2e} % provides \textsubscript
% use upquote if available, for straight quotes in verbatim environments
\IfFileExists{upquote.sty}{\usepackage{upquote}}{}
\ifnum 0\ifxetex 1\fi\ifluatex 1\fi=0 % if pdftex
  \usepackage[utf8]{inputenc}
\else % if luatex or xelatex
  \usepackage{fontspec}
  \ifxetex
    \usepackage{xltxtra,xunicode}
  \fi
  \defaultfontfeatures{Mapping=tex-text,Scale=MatchLowercase}
  \newcommand{\euro}{€}
\fi
% use microtype if available
\IfFileExists{microtype.sty}{\usepackage{microtype}}{}
\bibliographystyle{elsarticle-harv}
\ifxetex
  \usepackage[setpagesize=false, % page size defined by xetex
              unicode=false, % unicode breaks when used with xetex
              xetex]{hyperref}
\else
  \usepackage[unicode=true]{hyperref}
\fi
\hypersetup{breaklinks=true,
            bookmarks=true,
            pdfauthor={},
            pdftitle={RADAR Backscatter - can it be used to discriminate damages in forests?},
            colorlinks=false,
            urlcolor=blue,
            linkcolor=magenta,
            pdfborder={0 0 0}}
\urlstyle{same}  % don't use monospace font for urls

\setcounter{secnumdepth}{0}
% Pandoc toggle for numbering sections (defaults to be off)
\setcounter{secnumdepth}{0}


% Pandoc header



\begin{document}
\begin{frontmatter}

  \title{RADAR Backscatter - can it be used to discriminate damages in forests?}
    \author[CIRGEO Interdepartmental Research Center of Research of Geomatics]{Francesco Pirotti\corref{1}}
   \ead{francesco.pirotti@unipd.it} 
    \author[TESAF Department]{Francesco Pirotti\corref{2}}
   \ead{francesco.pirotti@unipd.it} 
      \address[CIRGEO Interdepartmental Research Center of Research of Geomatics]{Agripolis, University of Padova, Legnaro 35020, Italy}
    \address[TESAF Department]{University of Padova, Legnaro 35020, Italy}
      \cortext[1]{Corresponding Author}
    \cortext[2]{Equal contribution}
  
  \begin{abstract}
  This is the abstract. It consists of two paragraphs.
  \end{abstract}
  
 \end{frontmatter}

\emph{Text based on elsarticle sample manuscript, see
\url{http://www.elsevier.com/author-schemas/latex-instructions\#elsarticle}}

\hypertarget{introduction}{%
\section{Introduction}\label{introduction}}

In this work we present results from the analysis of time-series of
RADAR backscatter values over forests that were subject to heavy damage
due to extreme wind (VAIA storm - 28 October 2018). The hypothesis is
that, with due processing, RADAR backscatter can be used to detect and
assess damage of forest, under certain conditions.

Remote sensing for determining changes in the Earth surface is a common
application. Optical remote sensing if very much used, through
classification of imagery before and after events and successive
comparison, or also by difference of specific band transformations. It
is well known that the main drawback of optical imagery is weather and
atmospheric condition in general. This is not a problem if the damaged
area can be assessed without urgency, ie. waiting for a cloud-free
imagery or at least partially cloud-free areas is not a problem. If a
more timely assessment must be undertaken, then doing without the
atmospheric problem can become a big advantage. This is were active
remote sensing with micro-wave wavelength can bring a significant
advantage.

The effect of clouds over RADAR backscatter is negligible, but many
other factors concur to backscatter values and variations thereof .

Satellite open data allows a large audience of researchers to test
methods for interpreting phenomena impacting the Earth surface. Optical
and SAR data have complementary characteristics that can be integrated
to provide more information to models and methods. The Sentinel 1 C-band
SAR data {[}\protect\hyperlink{ref-Torres2012}{1}{]} and the Sentinel 2
optical multispectral data {[}\protect\hyperlink{ref-Drusch2012}{2}{]}
provide open data in the two realms of SAR and optical. The datasets are
also available as products in Google Earth Engine (GEE), an integrated
platform designed to empower not only traditional remote sensing
scientists but also a wider audience
{[}\protect\hyperlink{ref-Gorelick2017}{3}{]}. With the availability of
both optical and SAR data free of cost and already preprocessed, the
Copernicus datasets represent an optimal and rapid tool for the
assessment of forest damages, such as windthrows (\textbf{???} - not yet
published).

Early applications of remote sensing in forestry to detect and assess
damages dates to the 80's and was related to land-use/land-cover
classification using Thematic Mapper (TM), Landsat Multispectral Scanner
(MSS), the Advanced Very High Resolution Radiometer (AVHRR), and the
French Systeme Probatoire d'Observation de la Terre (SPOT) are passive
sensors applied to classification tasks
{[}\protect\hyperlink{ref-Iverson1989}{4}{]}.

Radar data started being used over forest environment to detect
deforestation, showing that stronger backscatter signals arrive at the
radar sensor from smooth, deforested areas
{[}\protect\hyperlink{ref-Saatchi1997}{5},\protect\hyperlink{ref-Stone1988}{6}{]}.
It must be noted that this requires the forest to be free from fallen
trunks, which is not true in the case windthrow were stems are left on
the ground.

Investigation on how forest parameters can be estimated with radar also
started in the 80's. A sensor with L-Band radar recording four
polarizations (VV, VH, HH, HV) and derived indices from their median
values can be used to estimate basal area, tree height, and total tree
biomass respectively with a correlation coefficient (r) of 0.50 0.82 and
0.77 {[}\protect\hyperlink{ref-Sader1987}{7}{]}.

{[}\protect\hyperlink{ref-VaglioLaurin2016}{8}{]} arrived to assess
biomass with radar and grey level Grey Level Co-occurrence Measure
(GLCM) texture features of NDVI reaching r=0.81.

With respect to windthrow, {[}\protect\hyperlink{ref-Ruetschi2019}{9}{]}
{[}\protect\hyperlink{ref-Farr_Rosen2007}{10}{]}

\hypertarget{material-and-methods}{%
\section{Material and methods}\label{material-and-methods}}

For both Sentinel 1 and PALSAR/PALSAR-2,

\hypertarget{theorycalculation}{%
\section{Theory/calculation}\label{theorycalculation}}

\hypertarget{sentinel-1}{%
\subsection{Sentinel 1}\label{sentinel-1}}

Google Earth Engine (GEE) provides Sentinel 1 in preprocessed GRD
products with \(\sigma^0\) (sigma-naught) of VV and VH polarizations,
after processing for removing thermal noise, calibrating radiometry and
converting \(\beta^0\) beta-naught to sigma-naught using a digital
elevation model (DEM). The DEM at the latitudes of the analysed study
areas used is from the Shuttle Radar Topography Mission (SRTM) that took
place in february 2000 \textbf{???}. Sigma-naught is provided in dB by
transformation the backscatter value \(Y=10*log_{10}(X)\) .
{[}\protect\hyperlink{ref-Small2011}{11}{]}. The GEE product was further
transformed to provide gamma-naught (\(\gamma^0\)) values, thus
correcting for the local incidence angle with the SRTM product. This
removed the bias between ascending and descenging orbits that was
evident from plotting the data.

Incidence angle was further corrected using a frequency-histogram based
mehod as described in {[}\protect\hyperlink{ref-Mladenova2013}{12}{]}.
This method is not site-specific or sensor-dependant. Is has also proven
to be effective not only for small incidence angles, which is the case
here as the area is over mountainous region.

\hypertarget{palsar-2}{%
\subsection{PALSAR-2}\label{palsar-2}}

Polarization data are stored in GEE as 16-bit digital numbers (DN). As
per indication in the GEE website, the DN values can be converted to
gamma naught (\(\gamma^0\)) values in decibel unit (dB) using the
following equation:

\[
\begin{aligned}
 \gamma^0 = 10*log_{10}(DN) - 83.0  
\end{aligned}
\]

where \(83.0 \  dB\) offset and \(\gamma^0\) is in dB.

values, thus correcting for the local incidence angle with the SRTM
product.

\hypertarget{results}{%
\section{Results}\label{results}}

\hypertarget{discussion}{%
\section{Discussion}\label{discussion}}

\hypertarget{conclusions}{%
\section{Conclusions}\label{conclusions}}

\hypertarget{acknowledgements}{%
\section{Acknowledgements}\label{acknowledgements}}

This effort is also part of the
\href{https://www.tesaf.unipd.it/ricerca/progetti-dip-tesaf}{VAIA FRONT
project - FRom lessong learned to future Options} .

\hypertarget{bibliography-styles}{%
\section{Bibliography styles}\label{bibliography-styles}}

There are various bibliography styles available. You can select the
style of your choice in the preamble of this document. These styles are
Elsevier styles based on standard styles like Harvard and Vancouver.
Please use BibTeX~to generate your bibliography and include DOIs
whenever available.

Here are two sample references: {[}{\textbf{???}}{]}.

\hypertarget{references}{%
\section*{References}\label{references}}
\addcontentsline{toc}{section}{References}

\hypertarget{refs}{}
\leavevmode\hypertarget{ref-Torres2012}{}%
{[}1{]} Torres R, Snoeij P, Geudtner D, Bibby D, Davidson M, Attema E,
et al. GMES Sentinel-1 mission. Remote Sensing of Environment 2012.
doi:\href{https://doi.org/10.1016/j.rse.2011.05.028}{10.1016/j.rse.2011.05.028}.

\leavevmode\hypertarget{ref-Drusch2012}{}%
{[}2{]} Drusch M, Del Bello U, Carlier S, Colin O, Fernandez V, Gascon
F, et al. Sentinel-2: ESA's Optical High-Resolution Mission for GMES
Operational Services. Remote Sensing of Environment 2012.
doi:\href{https://doi.org/10.1016/j.rse.2011.11.026}{10.1016/j.rse.2011.11.026}.

\leavevmode\hypertarget{ref-Gorelick2017}{}%
{[}3{]} Gorelick N, Hancher M, Dixon M, Ilyushchenko S, Thau D, Moore R.
Google Earth Engine: Planetary-scale geospatial analysis for everyone.
Remote Sensing of Environment 2017.
doi:\href{https://doi.org/10.1016/j.rse.2017.06.031}{10.1016/j.rse.2017.06.031}.

\leavevmode\hypertarget{ref-Iverson1989}{}%
{[}4{]} Iverson LR, Graham RL, Cook EA. Applications of satellite remote
sensing to forested ecosystems. Landscape Ecology 1989.
doi:\href{https://doi.org/10.1007/BF00131175}{10.1007/BF00131175}.

\leavevmode\hypertarget{ref-Saatchi1997}{}%
{[}5{]} Saatchi SS, Soares JV, Alves DS. Mapping deforestation and land
use in Amazon rainforest by using SIR-C imagery. Remote Sensing of
Environment 1997.
doi:\href{https://doi.org/10.1016/S0034-4257(96)00153-8}{10.1016/S0034-4257(96)00153-8}.

\leavevmode\hypertarget{ref-Stone1988}{}%
{[}6{]} Stone TA, Woodwell GM. Shuttle imaging radar a analysis of land
use in Amazonia. International Journal of Remote Sensing 1988.
doi:\href{https://doi.org/10.1080/01431168808954839}{10.1080/01431168808954839}.

\leavevmode\hypertarget{ref-Sader1987}{}%
{[}7{]} Sader SA, Wu S-T. Multipolarization SAR Data for Surface Feature
Delineation and Forest Vegetation Characterization. IEEE Transactions on
Geoscience and Remote Sensing 1987.

\leavevmode\hypertarget{ref-VaglioLaurin2016}{}%
{[}8{]} Vaglio Laurin G, Pirotti F, Callegari M, Chen Q, Cuozzo G,
Lingua E, et al. Potential of ALOS2 and NDVI to Estimate Forest
Above-Ground Biomass, and Comparison with Lidar-Derived Estimates.
Remote Sensing 2016;9:18.
doi:\href{https://doi.org/10.3390/rs9010018}{10.3390/rs9010018}.

\leavevmode\hypertarget{ref-Ruetschi2019}{}%
{[}9{]} Rüetschi M, Small D, Waser L. Rapid Detection of Windthrows
Using Sentinel-1 C-Band SAR Data. Remote Sensing 2019;11:115.
doi:\href{https://doi.org/10.3390/rs11020115}{10.3390/rs11020115}.

\leavevmode\hypertarget{ref-Farr_Rosen2007}{}%
{[}10{]} Farr TG, Rosen PA, Caro E, Crippen R, Duren R, Hensley S, et
al. The Shuttle Radar Topography Mission. Reviews of Geophysics 2007;45.
doi:\href{https://doi.org/10.1029/2005RG000183}{10.1029/2005RG000183}.

\leavevmode\hypertarget{ref-Small2011}{}%
{[}11{]} Small D. Flattening gamma: Radiometric terrain correction for
SAR imagery. IEEE Transactions on Geoscience and Remote Sensing 2011.
doi:\href{https://doi.org/10.1109/TGRS.2011.2120616}{10.1109/TGRS.2011.2120616}.

\leavevmode\hypertarget{ref-Mladenova2013}{}%
{[}12{]} Mladenova IE, Jackson TJ, Bindlish R, Hensley S. Incidence
Angle Normalization of Radar Backscatter Data. IEEE Transactions on
Geoscience and Remote Sensing 2013;51:1791--804.
doi:\href{https://doi.org/10.1109/TGRS.2012.2205264}{10.1109/TGRS.2012.2205264}.


\end{document}


